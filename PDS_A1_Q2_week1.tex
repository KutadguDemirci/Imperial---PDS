% Options for packages loaded elsewhere
\PassOptionsToPackage{unicode}{hyperref}
\PassOptionsToPackage{hyphens}{url}
\documentclass[
]{article}
\usepackage{xcolor}
\usepackage[margin=1in]{geometry}
\usepackage{amsmath,amssymb}
\setcounter{secnumdepth}{-\maxdimen} % remove section numbering
\usepackage{iftex}
\ifPDFTeX
  \usepackage[T1]{fontenc}
  \usepackage[utf8]{inputenc}
  \usepackage{textcomp} % provide euro and other symbols
\else % if luatex or xetex
  \usepackage{unicode-math} % this also loads fontspec
  \defaultfontfeatures{Scale=MatchLowercase}
  \defaultfontfeatures[\rmfamily]{Ligatures=TeX,Scale=1}
\fi
\usepackage{lmodern}
\ifPDFTeX\else
  % xetex/luatex font selection
\fi
% Use upquote if available, for straight quotes in verbatim environments
\IfFileExists{upquote.sty}{\usepackage{upquote}}{}
\IfFileExists{microtype.sty}{% use microtype if available
  \usepackage[]{microtype}
  \UseMicrotypeSet[protrusion]{basicmath} % disable protrusion for tt fonts
}{}
\makeatletter
\@ifundefined{KOMAClassName}{% if non-KOMA class
  \IfFileExists{parskip.sty}{%
    \usepackage{parskip}
  }{% else
    \setlength{\parindent}{0pt}
    \setlength{\parskip}{6pt plus 2pt minus 1pt}}
}{% if KOMA class
  \KOMAoptions{parskip=half}}
\makeatother
\usepackage{color}
\usepackage{fancyvrb}
\newcommand{\VerbBar}{|}
\newcommand{\VERB}{\Verb[commandchars=\\\{\}]}
\DefineVerbatimEnvironment{Highlighting}{Verbatim}{commandchars=\\\{\}}
% Add ',fontsize=\small' for more characters per line
\usepackage{framed}
\definecolor{shadecolor}{RGB}{248,248,248}
\newenvironment{Shaded}{\begin{snugshade}}{\end{snugshade}}
\newcommand{\AlertTok}[1]{\textcolor[rgb]{0.94,0.16,0.16}{#1}}
\newcommand{\AnnotationTok}[1]{\textcolor[rgb]{0.56,0.35,0.01}{\textbf{\textit{#1}}}}
\newcommand{\AttributeTok}[1]{\textcolor[rgb]{0.13,0.29,0.53}{#1}}
\newcommand{\BaseNTok}[1]{\textcolor[rgb]{0.00,0.00,0.81}{#1}}
\newcommand{\BuiltInTok}[1]{#1}
\newcommand{\CharTok}[1]{\textcolor[rgb]{0.31,0.60,0.02}{#1}}
\newcommand{\CommentTok}[1]{\textcolor[rgb]{0.56,0.35,0.01}{\textit{#1}}}
\newcommand{\CommentVarTok}[1]{\textcolor[rgb]{0.56,0.35,0.01}{\textbf{\textit{#1}}}}
\newcommand{\ConstantTok}[1]{\textcolor[rgb]{0.56,0.35,0.01}{#1}}
\newcommand{\ControlFlowTok}[1]{\textcolor[rgb]{0.13,0.29,0.53}{\textbf{#1}}}
\newcommand{\DataTypeTok}[1]{\textcolor[rgb]{0.13,0.29,0.53}{#1}}
\newcommand{\DecValTok}[1]{\textcolor[rgb]{0.00,0.00,0.81}{#1}}
\newcommand{\DocumentationTok}[1]{\textcolor[rgb]{0.56,0.35,0.01}{\textbf{\textit{#1}}}}
\newcommand{\ErrorTok}[1]{\textcolor[rgb]{0.64,0.00,0.00}{\textbf{#1}}}
\newcommand{\ExtensionTok}[1]{#1}
\newcommand{\FloatTok}[1]{\textcolor[rgb]{0.00,0.00,0.81}{#1}}
\newcommand{\FunctionTok}[1]{\textcolor[rgb]{0.13,0.29,0.53}{\textbf{#1}}}
\newcommand{\ImportTok}[1]{#1}
\newcommand{\InformationTok}[1]{\textcolor[rgb]{0.56,0.35,0.01}{\textbf{\textit{#1}}}}
\newcommand{\KeywordTok}[1]{\textcolor[rgb]{0.13,0.29,0.53}{\textbf{#1}}}
\newcommand{\NormalTok}[1]{#1}
\newcommand{\OperatorTok}[1]{\textcolor[rgb]{0.81,0.36,0.00}{\textbf{#1}}}
\newcommand{\OtherTok}[1]{\textcolor[rgb]{0.56,0.35,0.01}{#1}}
\newcommand{\PreprocessorTok}[1]{\textcolor[rgb]{0.56,0.35,0.01}{\textit{#1}}}
\newcommand{\RegionMarkerTok}[1]{#1}
\newcommand{\SpecialCharTok}[1]{\textcolor[rgb]{0.81,0.36,0.00}{\textbf{#1}}}
\newcommand{\SpecialStringTok}[1]{\textcolor[rgb]{0.31,0.60,0.02}{#1}}
\newcommand{\StringTok}[1]{\textcolor[rgb]{0.31,0.60,0.02}{#1}}
\newcommand{\VariableTok}[1]{\textcolor[rgb]{0.00,0.00,0.00}{#1}}
\newcommand{\VerbatimStringTok}[1]{\textcolor[rgb]{0.31,0.60,0.02}{#1}}
\newcommand{\WarningTok}[1]{\textcolor[rgb]{0.56,0.35,0.01}{\textbf{\textit{#1}}}}
\usepackage{graphicx}
\makeatletter
\newsavebox\pandoc@box
\newcommand*\pandocbounded[1]{% scales image to fit in text height/width
  \sbox\pandoc@box{#1}%
  \Gscale@div\@tempa{\textheight}{\dimexpr\ht\pandoc@box+\dp\pandoc@box\relax}%
  \Gscale@div\@tempb{\linewidth}{\wd\pandoc@box}%
  \ifdim\@tempb\p@<\@tempa\p@\let\@tempa\@tempb\fi% select the smaller of both
  \ifdim\@tempa\p@<\p@\scalebox{\@tempa}{\usebox\pandoc@box}%
  \else\usebox{\pandoc@box}%
  \fi%
}
% Set default figure placement to htbp
\def\fps@figure{htbp}
\makeatother
\setlength{\emergencystretch}{3em} % prevent overfull lines
\providecommand{\tightlist}{%
  \setlength{\itemsep}{0pt}\setlength{\parskip}{0pt}}
\usepackage{bookmark}
\IfFileExists{xurl.sty}{\usepackage{xurl}}{} % add URL line breaks if available
\urlstyle{same}
\hypersetup{
  pdftitle={Assessment 1 MATH70094 Programming for Data Science Autumn 2025 },
  hidelinks,
  pdfcreator={LaTeX via pandoc}}

\title{\normalsize Assessment 1 \hfill MATH70094 Programming for Data
Science \hfill Autumn 2025\\
\rule{\linewidth}{0.5mm}}
\author{}
\date{\vspace{-2.5em}}

\begin{document}
\maketitle

\subsection{Enter Kutadgu Gokalp Demirci -
06071571}\label{enter-kutadgu-gokalp-demirci---06071571}

\section{Question 2 (20 marks)}\label{question-2-20-marks}

You have heard that an experienced spy should not only be proficient in
Python, but also in R. Your task in this question is to provide R
statements and functions to encode and decode string messages, as well
as to do a simple analysis of a message. Some helpful
functions/techniques for this question are:

\begin{itemize}
\tightlist
\item
  cat, length, nchar, paste, strsplit, toupper, intToUtf8
\item
  sapply, grep, gsup
\item
  which, factor
\end{itemize}

It is easy to see how to use these functions with the R help by typing
\texttt{?length} in the R console. There are usually several ways to
answer each question, and sometimes you might not use all or even any of
the suggested functions.

\begin{Shaded}
\begin{Highlighting}[]
\CommentTok{\# Do not alter this}

\NormalTok{msg1 }\OtherTok{\textless{}{-}} \StringTok{"THE PASSWORD IS MY NAME"} 
\NormalTok{msg2 }\OtherTok{\textless{}{-}} \StringTok{"I AM A SPY WRITING A SPY MESSAGE"}
\NormalTok{msg3 }\OtherTok{\textless{}{-}} \StringTok{"XLMW*MW*XLI*HIVW*WLSYWI*UVG*MW*XLI*WWT*UIAG*OLKDL*XLI*XLMW*AU*AU*XLMW*MW*XLI*HIVW*WLSYWI*UVG*MW*XLI*WWT*UIAG*OLKDL*XLI*XLMW*AU*AU"}

\CommentTok{\# Do not alter this}
\end{Highlighting}
\end{Shaded}

\subsection{Code clarity (1 mark)}\label{code-clarity-1-mark}

There is a famous saying among software developers that code is read
more often than it is written. Marks will be awarded (or not awarded)
based on the clarity of the code and appropriate use of comments.

\subsection{Part A (2 marks)}\label{part-a-2-marks}

Repeat the task in Question 1, Part A, but now using R.

\subsection{Part B (8 marks)}\label{part-b-8-marks}

Repeat the task in Question 1, Part B, but now using R.

\subsection{Part C (2 marks)}\label{part-c-2-marks}

You have intercepted a message string stored in \texttt{msg2}. In order
to identify if this message is important your task is to write R
statements to locate the first occurrence (index) of the keyword
\texttt{"SPY"} in the message. Print this index preceded by an
informative print statement.

For example, if the message string is \texttt{"I\ AM\ A\ SPY"}, then you
should return

\texttt{The\ word\ index\ of\ the\ first\ occurrence\ of\ the\ keyword\ SPY\ is:\ 4.}

\subsection{Part D (4 marks)}\label{part-d-4-marks}

You have intercepted another encoded message string stored in
\texttt{msg3}. Your task is to write R statements to analyse the message
as follows:

\subsubsection{Part D(i) (2 marks)}\label{part-di-2-marks}

Count how often each letter \(A,B,..,Z\) appears in the message. Plot a
bar chart of the letter frequencies.

\subsubsection{Part D(ii) (2 marks)}\label{part-dii-2-marks}

Compute the length of each word in the message. Plot a histogram of the
word lengths.

\subsection{Part E (3 marks)}\label{part-e-3-marks}

Write an R function to generate a spy name from a person's first and
last names. The function should take as input a string with the person's
first name and a string with the person's last name, and return a string
with the spy name. The spy name is generated as follows:

\begin{itemize}
\tightlist
\item
  Convert the first and last names to uppercase.
\item
  Keep the first two letters of the first name, followed by the last two
  letters of the last name and conclude by a random two-digit number.
\end{itemize}

Print your spy name preceded by an informative print statement.

\end{document}
